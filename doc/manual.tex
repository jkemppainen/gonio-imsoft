\documentclass{article}

\begin{document}
\section{Introduction}

PupilImsoft is image acqusition software for Hamamatsu Orca Flash 4.0v3 and two Arduino connected rotary encoders.
PupilImsoft is to allow taking still images (static imaging) or a image time series while flashing stimulus light (dynamic imaging)
from a fixed sample orientation, while also saving the orientation (the two rotary encoders).

PupilImsoft runs on Windows but with small modifications (see TODO.txt) could be multiplatform.
The main problem is that if using Hamamatsu's Orca Flash as a camera, there may not be (official) drivers for other platforms.

\subsection{Work principles}

\subsection{Arduino and rotary encoders}
PupulImsoft enqueries the state of the two rotary encoders from an Arduino board.
Every time there's a change in encoders' values, Arduino sends the new values (angle pairs) to serial connection.


\subsection{Camera and NI boards}
There are currently two ways how the camera can be controlled:
Either by sending hardware trigger signals to the camera that has been set to acqusition mode by an external software
or by software using Micro-Manager's (an open source microscopy software) Python 2 bindings.

The first method, hardware triggering, is used in the static imaging mode because it offers very fast, low latency acqusition.
Every time a new angle pair is encountered, a hardware trigger signal is sent to the camera, and the camera acquires.
It is assumed that an external program sets the acqusition mode (Micro-Manager).


The second method, software acqusition, is used in the dynamic imaging.
Here, Micro-Manager's Python bindings are used to change camera's parameters, set aqusition on or take still images.
Because Micro-Manager by default provides Python 2 bindings and PupilImsoft is Python 3 code an extra server-client scheme is deployed.
Here, the camera server is ran in Python 2, and it recieves commands from the camera client over socket connection.

In dynamic imaging, stimulus-light-onset waits for a trigger from the camera, and then continues the protocol (Python code with sleep commands).
This is half-software half-hardware triggering method that seems to be accurate enough (Window's timer resolution is low, something from 1 ms to 16 ms, bad for realtime applications).


\subsection{NI boards}
Controlling National Instruments DAQ devices is done by National Instruments' nidaqmx Python package, that is an API to the NI-DAQmx driver.


\section{Usage}

Here, a short instruction how to run the experimental protocols is given.


\subsection{Static imaging}

To be completed...

\subsection{Dynamic imaging}

Set the rotary encoders to (0,0) initial positions, and confirm sample's orientation [FIGURE].
Launch IMSOFT.cmd shortcut.
Check that illumination light is focused on the specimen.
Confirm microscope settings.

To be completed...


\end{document}
